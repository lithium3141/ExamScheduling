\documentclass{article}

\usepackage{amsmath}
\usepackage{amsfonts}
\usepackage{array}
\usepackage{mdwlist}
\usepackage{wasysym}
\usepackage{fancyhdr}
\usepackage{graphicx}
\usepackage{minted}

\pagestyle{fancy}
\headheight 35pt
\begin{document}

\lhead{\textbf{MA444 \\ Project 1 \\ Exam Scheduling}}
\rhead{Tim Ekl \\ Christopher Gropp \\ Michael Hein}

\section{Introduction}

Exams are an unavoidable part of any institution of higher education, and with the wide variety of classes and unique mixture of students each quarter, finding a good schedule for finals can be a challenge. Each student has their own concept of what makes a ``good'' schedule, and while there tends to be consensus on most issues, it is generally impossible to satisfy all students.

\section{Problem}

Rose-Hulman provides 11 timeslots each quarter in which finals can take place; in each of these slots, up to 34 distinct classes can be accommodated. Given a list of classes, professors who teach them, and students that attend each class, we seek an optimal schedule that assigns each class (and all its students) to one of the available timeslots, without overusing any single slot or double-booking a professor or student.

\section{Assumptions}

To generate a schedule, we make several assumptions about the problem. First and foremost, there are certain constraints imposed by Rose-Hulman beyond the physical constraints of time and place; namely, administration requires that no student take three consecutive finals, including ``wraparound'' finals in which a student has a 6pm final one day and an 8am the next. In addition, we assume that no student can be assigned two finals in the same timeslot, despite the possibility of rescheduling one such final with the registrar or professors.

We then make a number of assumptions about what defines an optimal schedule. It is generally accepted that college students do not like to wake up early, and so we seek to avoid scheduling finals in the 8am timeslot whenever possible. In addition, many students make plans for breaks that begin soon after the quarter ends, and so we ``front-load'' the finals schedule and attempt to place many finals earlier in the week, though this effort takes lower priority than the time of day.

Finally, we define the concept of a ``conflict'' as a situation where a student is enrolled in multiple classes such that he or she prevents those classes from being scheduled in a certain way. Common examples include a student enrolled in two classes so that those classes may not have finals at the same time and a student enrolled in three classes so that those classes may not have finals scheduled consecutively.

\section{Algorithm}

We build a two-phase algorithm to create an optimal schedule. The first (primary) phase consists of a standard greedy constructive algorithm: we begin with no classes scheduled, and place the largest class into the most optimal timeslot as defined by our assumptions above. We continue in this manner, inserting classes into their best possible timeslots (generating no conflicts), until no further classes can be scheduled due to conflicts generated by the remainder of the student body.

At this point, we switch to the second phase of our algorithm: we remove a single class (chosen specifically to resolve the most conflicts) from the schedule and choose another, ``less optimal'' class to reschedule in its place. If this switch makes it possible to continue scheduling other classes, it is kept and the algorithm falls back to the first phase; otherwise, the change is reverted, as it provides fewer students with a valid finals schedule. This process continues until no class switches make it possible to schedule further.

\section{Discussion}

Using the algorithm described above, we are able to schedule all classes and still maintain all the constraints of the problem. In addition, we succeed in providing many students with finals early in the week or in the evenings; 44 sections of classes are scheduled for Monday, and the result tapers off to schedule only 9 sections for Thursday.

The algorithm runs in roughly $O(s)$ time, where $s$ is the number of students under consideration. Since it is primarily greedy and only ever backtracks one step, the longest-running portion of the algorithm is the consideration of students to check for conflicts and schedule a new class; the algorithm must look at all students in one class (a potentially large number) and compare them to the classes that have already been scheduled (generally a much smaller number).

\section{Conclusion}

We have found that while a pure greedy algorithm can produce desirable schedules for a large proportion of the student body, it does not necessarily guarantee that all students will have a workable finals schedule; almost 10\% of the students under consideration were not scheduled, a situation which would have to be resolved by hand (or by relaxing the constraints of the problem). We have also found that a small number of changes to a greedy algorithm can resolve these difficulties without significantly altering the greedy phase of the algorithm.

\end{document}